% This is the ADASS_template.tex LaTeX file, 26th August 2016.
% It is based on the ASP general author template file, but modified to reflect the specific
% requirements of the ADASS proceedings.
% Copyright 2014, Astronomical Society of the Pacific Conference Series
% Revision:  14 August 2014

% To compile, at the command line positioned at this folder, type:
% latex ADASS_template
% latex ADASS_template
% dvipdfm ADASS_template
% This will create a file called aspauthor.pdf.}

\documentclass[11pt,twoside]{article}

% Do NOT use ANY packages other than asp2014.
\usepackage{asp2014}

\aspSuppressVolSlug
\resetcounters

% References must all use BibTeX entries in a .bibfile.
% References must be cited in the text using \citet{} or \citep{}.
% Do not use \cite{}.
% See ManuscriptInstructions.pdf for more details
\bibliographystyle{asp2014}

% The ``markboth'' line sets up the running heads for the paper.
% 1 author: "Surname"
% 2 authors: "Surname1 and Surname2"
% 3 authors: "Surname1, Surname2, and Surname3"
% >3 authors: "Surname1 et al."
% Replace ``Short Title'' with the actual paper title, shortened if necessary.
% Use mixed case type for the shortened title
% Ensure shortened title does not cause an overfull hbox LaTeX error
% See ASPmanual2010.pdf 2.1.4  and ManuscriptInstructions.pdf for more details
\markboth{O'Mullane et al.}{BoF:Science Platforms}

\begin{document}

\title{Birds of a Feather session on Science platforms}

% Note the position of the comma between the author name and the
% affiliation number.
% Author names should be separated by commas.
% The final author should be preceded by "and".
% Affiliations should not be repeated across multiple \affil commands. If several
% authors share an affiliation this should be in a single \affil which can then
% be referenced for several author names.
% See ManuscriptInstructions.pdf and ASPmanual2010.pdf 3.1.4 for more details
\author{William~O'Mullane$^1$,
Megan Sosey$^2$,
Hassan Siddiqui$^3$,
Gregory~Dubois-Felsmann$^4$,
Gerard Lemson$^5$,
Christophe Arviset$^6$,
Mike Fitzpatrick$^7$,
Ivelina Momcheva$^2$,
Sebastien Fabbro$^8$,
Brian Major$^8$\\
\affil{$^1$Large Synoptic Survey Telescope, Tucson, AZ, USA; \email{womullan@lsst.org}}
\affil{$^2$Space Telescope Science Institute, Baltimore, MD, USA}
\affil{$^3$Vega for Gaia/ESAC, Madrid, Spain}
\affil{$^4$IPAC, California Institute of Technology, Pasadena, CA, USA}
\affil{$^5$The Johns Hopkins University, Baltimore, MD, USA}
\affil{$^6$ESAC Science Data Centre, Madrid, Spain}
\affil{$^7$NOAO, Tucson, AZ, USA}
\affil{$^8$CADC, NRC Herzberg, Victoria, BC, Canada}
}
%Michael~Wise$^3$,

% This section is for ADS Processing.  There must be one line per author.
\paperauthor{William~O'Mullane}{womullan@lsst.org}{}{LSST}{}{Tucson}{AZ}{85719}{USA}
% \paperauthor{Name}{EMAIL}{ORCID}{Institue}{Dept}{Town}{State}{Zip}{Country}
\paperauthor{Megan~Sosey}{}{}{STScI}{}{Baltimore}{MD}{}{USA}
\paperauthor{Hassan~Siddiqui}{}{European Space Agency}{ESAC}{}{Madrid}{}{}{Spain}
\paperauthor{Gregory~Dubois-Felsmann}{gpdf@ipac.caltech.edu}{}{California Institute of Technology}{IPAC}{Pasadena}{California}{91125}{USA}
\paperauthor{Gerard~Lemson}{}{}{Johns Hopkins University}{}{Baltimore}{MD}{}{USA}
\paperauthor{Christophe~Arviset}{Christophe.Arviset@esa.int}{0000-0002-2311-740X}{European Space Agency}{ESAC Science Data Centre}{Villanueva de la Canada}{Madrid}{28691}{Spain}
\paperauthor{Mike~Fitzpatrick}{}{}{NOAO}{}{Tucson}{AZ}{85719}{USA}
\paperauthor{Ivelina~Momcheva}{}{}{STScI}{}{Baltimore}{MD}{}{USA}
\paperauthor{Sebastian~Fabbro}{}{}{NRC Herzberg}{CADC}{Victoria}{BC}{}{Canada}
\paperauthor{Brian~Major}{}{}{NRC Herzberg}{CADC}{Victoria}{BC}{}{Canada}

\begin{abstract}
How users will interact with data in the future is always unclear.  Currently we see Jupyter Notebooks or JupyterLab emerging in many places as the way forward for one aspect of this.  This BoF explored some topics around providing an environment for doing science.

\end{abstract}

\section{Introduction}

It seems timely to consider how we might offer users a smoother experience as they move between data providers.
Current VO services allow one to send queries to multiple centres but in the notebook environment one may wish to do something more sophisticated.
Indeed many of us are working on some advancement beyond the current portal/query model.
We had a few short presentations which are all on GitHub\footnote{\url{https://github.com/lsst-dm/adass27-womullan/tree/master/presentations}} and are summarized here.

\paragraph*{LSST Approach (Dubois-Felsmann)} LSST \citep{2008arXiv0805.2366I} Science Platform \citep{LSE-319} will give access to the data and visualization tools and documentation, we see three aspects to this, the traditional Portal, the Web Services APIs and the Jupyter/Notebook.
It will allow collaboration and allow for added value processing close to the data using Jupyter. In principle you could write C++ or any other language in that system, it will come with the DM stack \citep{O3-1_adassxxv} pre-loaded.

\paragraph*{SciServer Approach (Lemson)} SciServer is format agnostic storage with extensible tools (query and analysis), it allows hosting and sharing datasets. Near data access is provided with Jupyter. Notebooks can be executed in batch mode but no MPI type processing is available at the moment.

\paragraph*{ESAC Science Data Centre (Arviset)} The Science Exploitation and Preservation Platform intends to provide a data computing environment close to the ESAC science archives, enabling scientists to run their code where the data reside, and share their results and code with other people. In addition, the platform will offer an environment to easily run legacy software from ESA space science missions.

\paragraph*{NOAO approach (Fitzpatrick)} Data Lab \citep{2016SPIE.9913E..0LF} provides full sky exploration of images/catalogs and approximately 1 PB for local user storage (including personal databases), it allows workflows to run close to the data and will soon offer data-publication services. Provides Jupyter notebooks and legacy code execution as containerized applications run in sync/async mode. User support includes docs, example notebooks, web forum and helpdesk contact; few users so these haven't been heavily exercised yet.  Open for new users but still in development.

\paragraph*{STScI DSMO (Momcheva)} We aim to increase science output from holdings, shorten turn around time, connect multi wavelength resources, and we are considering a Jupyter hub system deployed on Amazon. Cost would be relatively cheap, and some of that cost may be moved to users.

\paragraph*{CANFAR (Fabbro)} CANFAR has been providing user object-like storage through VOSpace, VM on demand, group management and user VM batch processing since 2010 \citep{2010SPIE.7740E..1IG}. It provides user support including documentation, help desk, and a slack channel. An increasing number of teams setup their own JupyterHub and container management on top of VMs. A proposal has been submitted to provide containerized workflows, tiered user storage with file-system access, user database-as-a-service, all integrated to a Jupyter based frontend.

\paragraph*{CADC/IVOA (Major)} The IVOA Grid and Web Services working group is working with the Knowledge Discovery interest group to define use cases for remote code execution.  This is recognized as a high priority item by the IVOA Committee for Science Priorities.  The goal is fast interoperable computing services close to the data, with support for Machine Learning and reproducability.  The implementation of a number of prototypes are in progress.  It has been recognized that the definition of a remote computing capability must allow the operator to optimize the execution.  The GWS working group hopes to make significant progress on Science Platforms for the spring 2018 interop meeting.  

\section{Discussion}

A few good principles were mentioned by Mike Fitzpatrick:
\begin{itemize}
\item multiple entry points into the system (web, notebooks, command line tools, scripting APIs);
\item language-agnostic (Python flask micro-services architecture, restful interface);
\item  enable user developed tools;
\item  established standards with hidden complexity for friendly interfaces;
\item  provide access to external data/services vs local ingest.
\end{itemize}


A few key questions were presented to seed the discussion:
\begin{itemize}
\item How many of us are using Jupyter (Lab/Hub/Notebooks) and how long will it last?
        \begin{itemize}
        \item How do we share notebooks?
        \item How can we make them (more) portable?
        \end{itemize}
\item How do we handle batch processing?
        \begin{itemize}
        \item Is Universal worker service sufficient?
        \item Quotas is this the new regime?
        \item How to handle single sign on and authorization/authentication?
        \end{itemize}

\end{itemize}


A show of hands by those doing a project like that described in the introduction above  showed roughly a 50-50 split of those using Jupyter as opposed to not using it (with LSST being the only users of JupyterLab). There were about 70 people in the room and perhaps 30 did not raise their hand for either option.

Security was discussed:
SciServer access is by account only but anyone can self register and there is no check on registrations.
Data Lab  is actually checking on people who apply for accounts, the APIs authenticate to services, although you can still run an anonymous notebook. The advantage of having an anonymous server is for tutorials and EPO projects.

What about authentication? Many use GitHub authentication. This is a concern, we do not wish to  force people to use any one of multiple authentication services. All agreed that this is not a good idea.
We will probably have to provide multiple authentication services, hopefully will get to an automatic way of authentication via API for science user validation.
You will still need an "account," the real question is how that is created.
For example, that could be done using your credentials from an existing academic organization.
Data rights can be complicated as well -- we did not delve in to that topic too far in this forum.


An NCSA research scientist (Nathan Goldbaum), mentioned there's an effort going on there, WHOLETALE\footnote{\url{http://wholetale.org/}},  which is a Docker stack that allows you to deploy on your own hardware. This uses Docker swarm as the backend balancing system. It is used as a platform for data sharing and computing on top of hardware the user controls.

There is a change in paradigm for data centres: providing data is different than providing processing time.
Do we need to consider proposals for a type of compute with I/O and sources, where  a committee then decides which proposals merit system time?
LSST is indeed thinking about these things for the future but wishes to see how things evolve first.
CADC say let's first get users, the community is not enormous, they have rough estimates of their needs and are not furtive.
So far these worries haven't been a problem. Resources essentially were not a problem on Skyserver/CasJobs either though they provided limited near data processing only.

It is important to open doors to reprocess images, going to cloud services gives a new user space to provide resources through other methods. Can the containerization run on multiple infrastructures, with great portability?
This seems to be where these things are heading!


How can we combine and connect all the resources in the future (the dream of the IVOA)?
How can we communicate between the large centers better?
That was the point of this BoF.
Maybe we should look at the channels others outside of the astronomy community have used?
At least in the USA there is the  national data services consortium, all sciences, had discussions there about containerization for mutually beneficial interactions, NDS has one or two meetings a year: \url{http://nationaldataservice.org}


The other interesting topic raised was reusing and bringing old code to the new platforms mentioned by NOAO and ESAC.
ESAC and NOAO are places with legacy processing software that needs specific machines to run on.
Porting CASA was complicated with GUIs, GUIs are in general problematic.
One way to preserve the software is to make a VM available and use VNC in the browser to access the GUI (not great but it does work).
STScI find it  important to educate our users on new and updated software instead of having them continue to run legacy code.
Also rather than move code to data, provide virtual storage space for use, you just move your data. Okay for small datasets, but helps you avoid rewriting code just to use the new interface.




\section{Conclusion}

Continue the conversation on SLACK at \url{https://scienceplatforms.slack.com/} or join the IVOA GWS list (\url{mailto:grid@ivoa.net}). There was a general consensus that a specific workshop on this topic may be useful early in 2018.

\acknowledgments Some of this material is based upon work supported in part by the National Science Foundation through Cooperative Agreement 1258333 managed by the Association of Universities for Research in Astronomy (AURA), and the Department of Energy under Contract No.\ DE-AC02-76SF00515 with the SLAC National Accelerator Laboratory. Additional LSST funding comes from private donations, grants to universities, and in-kind support from LSSTC Institutional Members.

\bibliography{B5}  % For BibTex

\end{document}
